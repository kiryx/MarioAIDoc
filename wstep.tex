\section{Wstęp teoretyczny}

\subsection{Historia sztucznej inteligencji.}
\begin{par}
Na początku lat 40 matematycy i inżynierowie z ośrodków badawczych zaczęli zastanawiać się nad możliwością stworzenia sztucznego mózgu.
Pierwsze formalne centrum badawcze pracujące nad zagadnieniem sztucznej inteligencji zostało powołane do życia w 1956 roku w Dartmouth College, 16 lat po wynalezieniu pierwszego programowalnego komputera.
Początkowo nazywane przedswięwzięciem stworzenia pierwszego ``Elektronicznego mózgu'' zostało poważnie potraktowane przez ówczesnych naukowców, i tworzyło dobre perspektywy dla ekonomistów i bankierów.
Wielu badaczy zapowiedziało stworzenie maszyn dorównujących inteligencją ludziom w niespełna kilka dekad.
Specjalnie na ten cel rząd amerykański oraz brytyjski przeznaczyły budżet rzędu milionów dolarów.
\end{par}
\begin{par}
Pierwsze prace nad sztuczną inteligencją skupiały się na odwzorowaniu realnej pracy ludzkiego mózgu - sieci neuronów.
Naukowcy tacy jak Norbert Wiener, Claude Shannon oraz Alan Turing opracowali pierwsze pomysły stworzenia elektronicznego mózgu.
Powstało pojęcie sieci neuronowych, mocno później rozwijane m.in. przez Marvina Minskiego w jego pracach przez następne 50 lat.
Pierwsze programy skupiające się na sztucznej inteligencji w grach powstały na początku lat 50. Christopher Strachey był autorem pierwszego programu grającego w Warcaby.
Pierwszy program szachowy został napisany przez Dietricha Prinza.
W tym samym okresie Alan Turing opublikował pierwsze prace dotyczące możliwości utworzenia maszyny dysponującej ludzką inteligencją.
Zdefiniował test pozwalający to zmierzyć, nazywany potem Testem Turinga.
\end{par}
\begin{par}
W końcu po wielu latach stało się oczywiste iż symulacja nawet najprostszych mechanizmów myślowych jest niezwykle trudna w realizacji, a nawet najszybsze ówczesne komputery nie były w stanie wygrać z człowiekiem w partii szachów. 
Ostatecznie dziedzinie sztucznej inteligencji odebrano nieco wiarygodności, a wcześniej zapowiadane maszyny przerastające inteligencją ludzi, trafiły spowrotem na półki science-fiction. W roku 1973, znaczna część funduszy przeznaczonych na rozwój sztucznej inteligencji została wstrzymana przez amerykański i brytyjski rząd.
Niemniej prace nad sztuczną inteligencją trwają do dziś, aczkolwiek są bardziej uszczegółowione w naturze problemów których dotyczą.
\end{par}


\subsection{Sztuczna w dzisiejszych zastosowaniach.}
\begin{par}
Oprócz realizacji zadań w dziedzinie kategoryzacji danych, oraz rozpoznawaniu mowy lub obrazu, spora część badań skupia się na realizacji systemów podejmujących dezycje w ściśle określonym środowisku gry.
Pozornie służą one jedynie dostarczaniu rozrywki w szeroko popularnych grach komputerowych, szybko można się przekonać iż wiele takich projektów jest później podstawą do stworzenia bardziej praktycznych systemów. Zmieniając jedynie definicję środowiska okazuje się iż można te same strategie zastosować np. w grze na giełdzie.
Pojedynek Garriego Kasparowa z programem szachowym Deep Blue przeszedł już na stałe do histori jako pierwsze starcie człowieka z maszyną w dziedzinie intelektu. 
Innym, dość nowym przykładem może być klaster komputerów Watson, który przez kilka tygodni konkurował z czołówką graczy teleturnieju Jeopardy, popularnym w USA od 1964 roku.
Obydwa projekty pokonały swoich ludzkich przeciwników, podnosząc tym samym nieco nadszarpnięty wizerunek sztucznej inteligencji.
\end{par}



\subsection{Podstawy algorytmów w grach.}
\begin{par}
Jednym z podstawowych przykładów zastosowania sztucznej inteligencji w grach komputerowych są gry logiczne. 
Podstawowym algorytmem stosowanym w projektowaniu sztucznej inteligencji jest algorytm minmax. Dotyczy on gier o sumie stałej -- proporcja zysku jednego z graczy do straty drugiego jest stała. 
Najprostszym przykładem jest gra ``Kółko i krzyżyk'' gdzie przeszukiwane jest rekurencyjne drzewo wywołań kolejno sprawdzające wszystkie możliwe stany gry. 
W dowolnym momencie gry możemy przeanalizować wszystkie możliwe posunięcia każdego z graczy. 
Gra może zakończyć się remisem, zwycięstwem gracza A, bądź zwycięstwem gracza B.
Optymalizując decyzję gracza A, musimy kolejno sprawdzać możliwe posunięcia na planszy i reakcje gracza B. 
Dla każdego z nich możemy wówczas przeanalizować optymalną strategię dla gracza B (ponieważ zakładamy że do takiej będzie on dążył) i starać się znaleźć najlepszą ścieżkę która prowadzi do zwycięstwa gracza A, bądź remisu.
Zatem na początku gry podjęciu pierwszej decyzji musimy przeanalizować $9*8*7*..*1 = 9!$ ruchów.
Łatwo zauważyć iż algorytm taki ma ogromną złożoność i jedynie dla małych gier takich jak kółko i krzyżyk daje wynik w realnym czasie.
Bez optymalizacji, obraz brania pod uwagę symetrii planszy, daje to drzewo wywołań składające się 9! węzłów (łącznie z liśćmi). 
Oczywiście algorytm można zoptymalizować chociażby poprzez programowanie dynamiczne, lecz dla gier bardziej złożonych nie będziemy w stanie przeanalizować wszystkich możliwych sytuacji w grze w realnym czasie.
Przez lata szachy były jedną z gier niemożliwych do rozwiązania za pomocą powyższego podejścia.
Nawet dziś najlepsze programy szachowe nie grają idealnie - nie analizują wszystkich możliwych ruchów a jedynie kilkadziesiąt ruchów w przód.
Przy zastosowaniu optymalizacji oraz bazy danych zawierającej wiele strategii szachowych, współczesne programy szachowe wygrywają z najlepszymi graczami.
Nieco inna sytuacja jest w grze GO, gdzie plansza rozmiaru 19x19 stanowi nielada wyzwanie nawet dla współczesnych superkomputerów. 
Do dziś nie stworzono programu który wygrywałby z profesjonalnymi zawodnikami GO.
\end{par}
\begin{par}
Większość gier daje się sklasyfikwać ze względu na czas:
\begin{itemize}
	\item Gry turowe. Gracze naprzemiennie wykonują ruchy, przy czym czas na podjęcie decysji jest relatywnie duży - nawet do 10 sekund. Wiele nieskomplikowanych gier zostało już dawno rozwiązanych przez algorytmy typu minmax, do tego stopnia, że systemy grają w nie już niemal bezbłędnie. W wielu przypadkach przestrzeń rozwiązań jest jednak wciąż zbyt duża aby zrealizować to algorytmem dokładnym - wyżej wspomniana gra GO.
	\item Gry czasu rzeczywistego. Gra toczy się w dynamicznym środowisku gry, często z wieloma obiektami/graczami na raz. Często czas na podjęcie optymalnej decyzji przez algorytm jest mocno ograniczony - często nalezy podejmować nawet do 30 razy w ciągu sekundy. Oprócz tego próba dyskretyzacji środowiska i znalezienia najlepszego rozwiązania z przestrzeni stanów gry jest w praktyce niewykonalne. Przykładem mogą być różnego rodzaju trójwymiarowe gry akcji które często posiadają złożone środowiska gry, oraz wiele dynamicznych obiektów oraz graczy uczestniczących w rozgrywce.
\end{itemize}
Sztuczna inteligencja w grach może dotyczyć różnych aspektów gry. 
W grach logicznych (turowych) głównym, i jedynym problemem jest podjęcie najlepszej decyzji dla aktualnego stanu gry, prowadzącej do zwycięstwa. 
W większości gier logicznych sztuczna inteligencja ma za zadanie symulację godnego przeciwnika dla człowieka.
Często jednak te same algorytmy mogą służyć do celów edukacyjnych bądź do podpowiedzi - ten sam system grający w szachy może grać przeciwko nam, jak i podpowiadać nam ruchy na podstawie naszej pozycji na planszy.
W grach zręcznościowych oraz akcji (gry czasu rzeczywistego) występuje często inny rodzaj sztucznej inteligencji.
Ponieważ przestrzeń rozwiązań jest bardzo duża, często podjęcie decyzji może być wspomagane przez algorytm niedeterministyczny, bądź oparty na algorytmach genetycznych.
Algorytm minmax w większości przypadków zawodzi, bądź jego czas działania jest zbyt wolny do zastosowania w dynamicznym środowisku gry.
Korzystając z algorytmów ``jedynie'' optymalizujących rozgrywkę tracimy możliwość zasymulowania idealnej partii gry, jednak często wystarcza to dla osiągnięcia celu końcowego.
\end{par}



\subsection{Podstawy algorytmów genetycznych.}
\begin{par}
Często wykorzystywanym sposobem realizacji celu są algorytmy genetyczne.
Opierają się one na prawach ewolucji odkrytych przez Charlesa Darwina, i wzorują się na faktycznych rozwiązaniach doboru naturalnego występujących w przyrodzie.
Algorytm ewolucyjny opiera się na wprowadzeniu losowego czynnika do całej procedury, i tym też różni się od poprzedniego podejścia, iż jest niederministyczny. 
Początkowy chaos z czasem jest zastępowana przez odpowiednio przystosowaną populację rozwiązań, o ile zasymulujemy jej ewolucję dostateczną ilość razy. 
W większości algorytmów genetycznych można wydzielić kilka koniecznych do zaprojektowania klas bądź procedur.
\begin{enumerate}
\item Chromosom oraz Populacja
	\begin{par}
		Pierwszym krokiem jest zdefiniowanie typu danych odpowiednich do przetrzymywania informacji o danym osobniku.
		Odpowiednio zaprojektowany format danych (zwany Chromosomem) pozwoli na łatwą implementację pozostałych elementów oraz zapewni generowanie optymalnych wyników.
		Informacja ta często jest reprezentowana przez tablicę wartości, bądź listę cech przypisanych do danej klasy. 
		Chromosom odpowiada za informację o pojedynczym osobniku, natomiast Populacja traktowana jest jako wszystkie osobniki należące do danego zbioru w danej iteracji algorytmu. 
		O ile w podstawowych algorytmach genetycznych Populacja jest jedynie kontenerem, dobrze jest pamiętać o ewentualnym rozbudowaniu Populacji do bardziej złożonej klasy, dzięki czemu będziemy mieli możliwość prostego porównywania, bądź zapamiętywania całych populacji.
	\end{par}
\item Funkcja Przystosowania
	\begin{par}
		Kolejnym istotnym krokiem jest zdefiniowanie funkcji przystosowania.
		W doborze naturalnym występującym w przyrodzie, osobniki danego gatunku rośliny bądź zwierzęcia różnią się pod względem genetycznym. 
		Można wówczas wywnioskować iż część z nich jest lepiej przystosowana do danego środowiska, co z kolei wpływa na ich szanse przeżycia w trudnych sytuacjach, liczność potomstwa, długość życia.
		Ponieważ potomstwo dziedziczy geny po swoich rodzicach, ``zwycięskie'' cechy w kolejnym pokoleniu są bardziej powszechne.
		Odpowiednikiem funkcji przystosowania jest właśnie wynikowa cech danego osobnika która określa prawdopodobieństwo przekazania jego genów w kolejnym pokoleniu.
		Funkcja przystosowania jest dość prosta w realizacji, o ile dane dotyczące osobnika są łatwe do zmierzenia -- wówczas może być to jedynie kwestia policzenia wartości funkcji liniowej z odpowiednimi wagami, gdzie argumentami są wyniki osobnika podczas symulacji w środowisku.
		Mimo to w większości algorytmów genetycznych dobranie odpowiednich wag w funkcj przystosowania jest kluczowym czynnikiem nad którym później można długo pracować przy optymalizacji algorytmu.
	\end{par}
\item Krzyżowanie
	\begin{par}
		Po każdym kroku algorytmu zazwyczaj możemy uporządkować osobniki należące do bieżącej populacji i wylosować z niej pewien zbiór osobników najlepiej przystosowanych (wpływ na to wynik funkcji przystosowania). 
		Wówczas dokonujemy krzyżowania pomiędzy nimi, dzięki czemu otrzymujemy osobniki nowe, jednak posiadające pewne cechy swoich ``rodziców''.
		Krok ten jest kluczowy jeśli chcemy osiągać coraz lepsze wyniki w kolejnych populacjach, ponieważ od dobrej metody krzyżowania zależy czy kolejne populacje będą lepiej przystosowane do rozwiązania problemu.
		Złe zaprojektowanie krzyżowania jest jednym z częstszych powodów osiągania przez populację złych wyników, zwłaszcza gdy Chromosom jest złożony.
		Samo krzyżowanie również zazwyczaj posiada czynnik losowy (w klasycznych przykładach dotyczących krzyżowania się dwóch ciągów bitowych, losowany jest punkt łączenia się dwóch ciągów).
	\end{par}
\item Mutacja
	\begin{par}
		O ile początkowa losowość algorytmu polegająca na wylosowaniu pierwszej populacji jest szybko zastępowana przez populację osiągającą lepsze wyniki, 
		warto w trakcie całego procesu próbować modyfikować kilka osobników, nawet jeśli mogłoby to spowodować chwilowe pogorszenie populacji. 
		W innym przypadku zbyt uporządkowana procedura selecji i krzyżowania osobników spowoduje stagnację populacji.
		Często można to zauważyć gdy po kilku iteracjach większość, bądź cała populacja jest identyczna.
		Najczęstszą realizacją mutacji jest zmiana jakiegoś parametru (bądź grupy parametrów)danego osobnika na wartość nieco inną lecz podobną, bądź zupełnie losową.
		Ponieważ w dużej mierze zalezy to od budowy Chromosomu, nie ma uniwersalnej metody na zaimplementowanie mutacji.
		Najczęściej mutacja występuje z niskim prawdopodobieństwem:
		\begin{center}
			$p_m < 0.1$
		\end{center}
		Tak aby nie ingerować zbyt mocno w algorytm. Ostatecznie należy dążyć do pewnej systematycznej optymalizacji, a nie tylko polegać na czynniku losowym.
	\end{par}
\item Metoda Selekcji
	\begin{par}
		Sama metoda wyboru populacji rodzicielskiej również ma znaczenie, ponieważ jednak jest ona oparta na wartości fukcji przystosowania, to już sama metoda wyboru ma mniej krytyczne znaczenie.
		Najbardziej popularne metody selekcji to:
		\begin{enumerate}
			\item Metoda koła ruletki.
				\begin{par}
					Sama nazwa bierze się od popularnej gry w ruletkę, w której pole powierzchni każdego wycinka koła jest proporcjonalne do prawdopodobieństwa wylosowania danej liczby. 
					Oczywiście w klasycznej ruletce pola wycinków koła są równe, zatem szansa wylosowania każdej liczby jest taka sama.
					W samym algorytmie wirtualne ``wycinki koła'' nie muszą oczywiście być równe. 
					Osobnik który osiąga lepsze wyniki w funkcji przystosowania otrzymuje większe prawdopodobieństwo włączenia do populacji rodzicielskiej niż osobniki słabsze. 
					Aby to zrealizować losowana jest pewna wartość która potem jednoznacznie określa który osobnik został wylosowany.
					Praktycznie realizowane jest to w następujący sposób:
					\begin{center}
						$p(k)=\frac{f(k)}{\displaystyle\sum\limits_{i=0}^n f(i)}$
					\end{center}
					gdzie $p(k)$ oznacza prawdopodobieństwo wylosowana k-tego osobnika z populacji, a $f(i)$ wartość funkcji przystosowania i-tego osobnika.
				\end{par}
			\item Metoda rankingowa.
				\begin{par}
					W tej metodzie sortujemy osobniki malejąco względem funkcji przystosowania i wybieramy populację rodziców jako $m$ pierwszych osobników. 
					Ma to pewną wadę, gdyż powoduje po pewnym czasie stagnację (brak czynnika losowego). 
					Innym wariantem jest selekcja turniejowa w której najpierw dzielimy grupę na $G$ podgrup spośród których wybieramy najlepsze osobniki do populacji rodzicielskiej. 
					Otrzymujemy w ten sposób G rodziców, wśród których niekoniecznie są najlepsze osobniki globalnie (nawet z bardzo silnej grupy przechodzi tylko jeden osobnik). 
					Daje nam to już pewną losowość w wyborze populacji rodzicielskiej.
				\end{par}
			\item Połączenie kilku metod.
				\begin{par}
					Dodatkowym elementem może być połączenie kilku metod selekcji celem otrzymania najbardziej optymalnej selekcji dla danego problemu genetycznego. 
					W zasadzie bardziej złożone problemy ewolucyjne wręcz wymagają własnej inwencji do zaprojektowania dobrego systemu.
				\end{par}
		\end{enumerate}
	\end{par}
\end{enumerate}
	Dużą częścią dobrego systemu genetycznego jest odpowiednia możliwość konfiguracji danych odpowiadających za każdy z kroków.
	Mamy dzięki temu możliwość przetestowania różnych podejść do danego problemu bez bezpośrednich i często uciążliwych zmian w kodzie programu.
	Oprócz tego cały proces można zautomatyzować, dzięki czemu możemy w prosty sposób przetestować algorytm dla różnych danych konfiguracyjnych.
\end{par}
