\begin{par}
Jednym z prostszych przykładów zastosowania sztucznej inteligencji w grach komputerowych są gry logiczne. 
Podstawowym algorytmem stosowanym w projektowaniu sztucznej inteligencji jest algorytm minmax. Dotyczy on gier o sumie stałej -- proporcja zysku jednego z graczy do straty drugiego jest stała. 
Najprostszym przykładem jest gra ``Kółko i krzyżyk'' gdzie budowane jest rekurencyjne drzewo wywołań kolejno generujące wszystkie możliwe stany gry. 
W dowolnym momencie gry możemy przeanalizować wszystkie możliwe posunięcia graczy. Gra może zakończyć się remisem, zwycięstwem gracza A, bądź zwycięstwem gracza B.
Optymalizując decyzję gracza A, musimy kolejno sprawdzać możliwe posunięcia na planszy. Dla każdego z nich możemy wówczas przeanalizować optymalną strategię dla gracza B (ponieważ zakładamy że do takiej będzie on dążył) i starać się znaleźć najlepszą ścieżkę która prowadzi do zwycięstwa gracza A.
Zatem na początku gry podjęciu pierwszej decyzji musimy przeanalizować $9*8*7*..*1 = 9!$ ruchów.
Łatwo zauważyć iż algorytm taki ma ogromną złożoność i jedynie dla małych gier takich jak kółko i krzyżyk daje wynik w realnym czasie.
Bez optymalizacji, obraz brania pod uwagę symetrii planszy, daje to drzewo wywołań składające się 9! węzłów (łącznie z liśćmi). 
Oczywiście algorytm można zoptymalizować chociażby poprzez programowanie dynamiczne, lecz dla gier bardziej złożonych nie będziemy w stanie przeanalizować wszystkich możliwych sytuacji w grze w realnym czasie.
\end{par}
\begin{par}
Przez lata szachy były jedną z gier niemożliwych do rozwiązania za pomocą powyższego podejścia.
Nawet dziś najlepsze programy szachowe nie analizują wszystkich możliwych ruchów a jedynie kilkadziesiąt ruchów w przód.
Przy zastosowaniu optymalizacji oraz bazy danych zawierającej wiele strategii szachowych, współczesne programy szachowe wygrywają z najlepszymi graczami.
Nieco inna sytuacja jest w grze GO, gdzie plansza rozmiaru 19x19 stanowi nielada wyzwanie nawet dla współczesnych superkomputerów. 
Do dziś nie stworzono programu który wygrywałby z profesjonalnymi zawodnikami GO.
\end{par}
\begin{par}
Sztuczna inteligencja w grach może dotyczyć różnych aspektów gry. W grach logicznych głównym, i jedynym problemem jest podjęcie najlepszej decyzji dla aktualnego stanu gry, prowadzącej do zwycięstwa. 
W grach akcji występuje nieco inny rodzaj sztucznej inteligencji, skupiający się na symulowaniu gry żywego przeciwnika. 
Większość obliczeń składa się wówczas nie tyle na rozwiązaniu gry, co na zbliżeniu jej do poziomu ludzkich graczy. 
W wielu przypadkach, szczególnie grach wymagających od gracza dobrego refleksu, możliwe jest zrealizowanie algorytmu który bez problemu wygrywałby z każdym człowiekiem.
Mamy wówczas mamy do czynienia z operującymi w gre odbywającej się w czasie rzeczywistym. 
Niejednokrotnie w dzisiejszych grach informacje o aktualnym stanie gry są niemożliwe do przeanalizowania ze względu na ich ogrom, oraz ograniczony czas -- decyzja musi zostać podjęta np. 30 razy w ciągu sekundy. Klasyczne podejście minmax zawodzi lub jest wręcz nie do zrealizowania dla większości współczesnych gier platformowych lub akcji.
\end{par}
