\begin{par}
Oprócz realizacji zadań w dziedzinie kategoryzacji danych, oraz rozpoznawaniu mowy lub obrazu, spora część badań skupia się na realizacji systemów podejmujących dezycje w ściśle określonym środowisku gry.
Pozornie służą one jedynie dostarczaniu rozrywki w szeroko popularnych grach komputerowych, szybko można się przekonać iż wiele takich projektów jest później podstawą do stworzenia bardziej praktycznych systemów. Zmieniając jedynie definicję środowiska okazuje się iż można te same strategie zastosować np. w grze na giełdzie.
Pojedynek Garriego Kasparowa z programem szachowym Deep Blue przeszedł już na stałe do histori jako pierwsze starcie człowieka z maszyną w dziedzinie intelektu. 
Innym, dość nowym przykładem może być klaster komputerów Watson, który przez kilka tygodni konkurował z czołówką graczy teleturnieju Jeopardy, popularnym w USA od 1964 roku.
Obydwa projekty pokonały swoich ludzkich przeciwników, podnosząc tymsamym nieco nadszarpnięty wizerunek sztucznej inteligencji.
\end{par}
