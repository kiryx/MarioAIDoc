\begin{par}
Na początku lat 40 matematycy i inżynierowie z ośrodków badawczych zaczęli dyskutować nad możliwością stworzenia sztucznego mózgu.
Pierwsze formalne środowisko badawcze pracujące nad zagadnieniem sztucznej inteligencji zostało powołane do życia w 1956 roku w Dartmouth College, 16 lat po wynalezieniu pierwszego programowalnego komputera.
Początkowo nazywane przedswięwzięciem stworzenia pierwszego ``Elektronicznego mózgu'' zostało poważnie potraktowane przez ówczesnych naukowców, i tworzyło dobre perspektywy dla ekonomistów i bankierów.
Wielu badaczy zapowiedziało stworzenie maszyn dorównujących inteligencją ludziom w niespełna kilka dekad.
Specjalnie na ten cel rząd amerykański oraz brytyjski utworzyły budżety rzędu milionów dolarów.
\end{par}
\begin{par}
Pierwsze prace nad sztuczną inteligencją skupiały się na odwzorowaniu realnej pracy ludzkiego mózgu - sieci neuronów.
Naukowcy tacy jak Norbert Wiener, Claude Shannon oraz Alan Turing opracowali pierwsze pomysły stworzenia elektronicznego mózgu.
Powstało pojęcie sieci neuronowych, mocno później rozwijane przez Marvina Minskiego w jego pracach przez następne 50 lat.
Pierwsze programy skupiające się na sztucznej inteligencji w grach powstały na początku lat 50. Christopher Strachey był autorem pierwszego programu grającego w Warcaby.
Pierwszy program szachowy został napisany przez Dietricha Prinza.
W tym samym okresie Alan Turing opublikował pierwsze prace dotyczące możliwości utworzenia maszyny dysponującej ludzką inteligencją.
Zdefiniował test pozwalający to zmierzyć, nazywany potem Testem Turinga.
\end{par}
\begin{par}
W końcu po wielu latach stało się oczywiste iż symulacja nawet najprostszych mechanizmów myślowych jest niezwykle trudna w realizacji, a nawet najszybsze ówczesne komputery nie były w stanie wygrać z człowiekiem w partii szachów. 
Ostatecznie dziedzinie sztucznej inteligencji odebrano nieco wiarygodności, a wcześniej zapowiadane maszyny przerastające inteligencją ludzi, trafiły spowrotem na półki science-fiction. W roku 1973, znaczna część funduszy przeznaczonych na rozwój sztucznej inteligencji została wstrzymana przez amerykański i brytyjski rząd.
Niemniej prace nad sztuczną inteligencją trwają do dziś, aczkolwiek są bardziej uszczegółowione w naturze problemów których dotyczą.
\end{par}
