\begin{par}
Innym rodzajem sztucznej inteligencji są algorytmy ewolucyjne.
Opierają się one na prawach ewolucji odkrytych przez Charlesa Darwina, i wzorują się na faktycznych rozwiązaniach doboru naturalnego występującego w przyrodzie.
Algorytm ewolucyjny opiera się na wprowadzeniu losowego czynnika do całej procedury, i tym też różni się od poprzedniego podejścia, iż jest niederministyczny. 
Chaotyczność z czasem jest zastępowana przez odpowiednio przystosowaną populację, o ile zasymulujemy jej ewolucję dostateczną ilość razy. 
W większości algorytmów genetycznych można wydzielić kilka koniecznych do zaprojektowania klas bądź procedur.
\begin{enumerate}
\item Chromosom oraz Populacja
	\begin{par}
		Pierwszym krokiem jest zdefiniowanie typu danych odpowiednich do przetrzymywania informacji o danym osobniku.
		Odpowiednio zaprojektowany format danych (zwany Chromosomem) pozwoli na łatwą implementację pozostałych elementów oraz zapewni generowanie optymalnych wyników.
		Informacja ta często jest reprezentowana przez tablicę wartości, bądź listę cech przypisanych do danej klasy. 
		Chromosom odpowiada za informację o pojedynczym osobniku, natomiast Populacja traktowana jest jako wszystkie osobniki należące do danego zbioru w danej iteracji algorytmu. O ile w podstawowych algorytmach genetycznych Populacja jest jedynie kontenerem, dobrze jest pamiętać o ewentualnym rozbudowaniu Populacji do bardziej złożonej klasy, dzięki czemu będziemy mieli możliwość prostego porównywania, bądź zapamiętywania całych populacji.
	\end{par}
\item Funkcja Przystosowania
	\begin{par}
		Kolejnym istotnym krokiem jest zdefiniowanie funkcji przystosowania.
		W doborze naturalnym występującym w przyrodzie osobniki danego gatunku rośliny bądź zwierzęcia różnią się pod względem genetycznym. 
		Można wówczas wywnioskować iż część z nich jest lepiej przystosowana do danego środowiska, co z kolei wpływa na ich szanse przeżycia w trudnych sytuacjach, liczność potomstwa, długość życia.
		Ponieważ potomstwo dziedziczy geny po swoich rodzicach, ``zwycięskie'' cechy w kolejnym pokoleniu są bardziej powszechne.
		Odpowiednikiem funkcji przystosowania jest właśnie wynikowa cech danego osobnika która określa prawdopodobieństwo przekazania jego genów w kolejnym pokoleniu.
		Funkcja przystosowania jest dość prosta w realizacji, o ile dane dotyczące osobnika są łatwe do zmierzenia -- wówczas może być to jedynie kwestia policzenia wartości funkcji liniowej z odpowiednimi wagami.
		Mimo to w większości algorytmów genetycznych dobranie odpowiednich wag w funkcj przystosowania jest kluczowym czynnikiem nad którym później można długo pracować przy optymalizacji algorytmu.
	\end{par}
\item Krzyżowanie
	\begin{par}
		Przy przechodzeniu każdego kroku algorytmu zazwyczaj należy wylosować z populacji pewien zbiór osobników (wpływ na to wynik funkcji przystosowania) i dokonać krzyżowania pomiędzy nimi. 
		Krok ten jest kluczowy jeśli chcemy osiągać coraz lepsze wyniki w kolejnych populacjach, ponieważ od dobrej metody krzyżowania zależy czy kolejne osobniki będą lepiej przystosowane do rozwiązania problemu.
		Złe zaprojektowanie krzyżowania jest jednym z częstszych powodów osiągania przez populację złych wyników, zwłaszcza gdy Chromosom jest złożony.
		Oprócz tego samo krzyżowanie również zazwyczaj posiada czynnik losowy (w klasycznych przykładach dotyczących krzyżowania się dwóch ciągów bitowych, losowany jest punkt łączenia się dwóch ciągów).
	\end{par}
\item Mutacja
	\begin{par}
		O ile początkowa losowość algorytmu polegająca na wylosowaniu pierwszej populacji jest szybko zastępowana przez populację osiągającą lepsze wyniki, 
		warto w trakcie całego procesu próbować modyfikować kilka osobników, nawet jeśli mogłoby to spowodować chwilowe pogorszenie populacji. 
		W innym przypadku zbyt uporządkowana procedura selecji i krzyżowania osobników spowoduje stagnację populacji. 
		Często można to zauważyć gdy po kilku iteracjach większość, bądź cała populacja jest identyczna.
		Najczęstszą realizacją mutacji jest zmiana jakiegoś parametru osobnika, bądź zamiana go na losową wartość.
		Ponieważ w dużej mierze zalezy to od budowy Chromosomu, nie ma uniwersalnej metody na zaimplementowanie mutacji.
		Najczęściej mutacja występuje z niskim prawdopodobieństwem:
		\begin{center}
			$p_m < 0.1$
		\end{center}
	\end{par}
\item Metoda Selekcji
	\begin{par}
		Sama metoda wyboru populacji rodzicielskiej również ma znaczenie, o ile jednak jest ona oparta na wartości fukcji przystosowania, to już sama metoda wyboru ma mniej krytyczne znaczenie.
		Najbardziej popularne metody selekcji to:
		\begin{enumerate}
			\item Metoda koła ruletki.
				\begin{par}
					Sama nazwa bierze się od gry w ruletkę, w której pole powierzchni każdego wycinka koła jest proporcjonalne do prawdopodobieństwa wylosowania danej liczby. 
					Oczywiście w klasycznej ruletce pola wycinków koła są równe, zatem szansa wylosowania każdej liczby jest taka sama.
					W samym algorytmie wirtualne ``wycinki koła'' nie muszą oczywiście być równe. 
					Osobnik który lepiej wypada w funkcji przystosowania otrzymuje większe pole niż osobniki słabsze. 
					Następnie losowana jest pewna wartość która jednoznacznie określa który osobnik został wylosowany.
					Praktycznie realizowane jest to w następujący sposób:
					\begin{center}
						$p(k)=\frac{f(k)}{\displaystyle\sum\limits_{i=0}^n f(i)}$
					\end{center}
					gdzie $p(k)$ oznacza prawdopodobieństwo wylosowana k-tego osobnika z populacji, a $f(i)$ wartość funkcji przystosowania i-tego osobnika
				\end{par}
			\item Metoda rankingowa.
				\begin{par}
					W tej metodzie sortujemy osobniki malejąco względem funkcji przystosowania i wybieramy populację rodziców jako $m$ pierwszych osobników. 
					Ma to pewną wadę, gdyż powoduje po pewnym czasie stagnację (brak czynnika losowego). 
					Innym wariantem jest selekcja turniejowa w której najpierw dzielimy grupę na $G$ podgrup spośród których wybieramy najlepsze osobniki do populacji rodzicielskiej. 
					Otrzymujemy w ten sposób G rodziców, wśród których niekoniecznie są najlepsze osobniki globalnie (nawet z bardzo silnej grupy przechodzi tylko jeden osobnik). 
					Daje nam to już pewną losowość w wyborze populacji rodzicielskiej.
				\end{par}
			\item Połączenie kilku metod.
				\begin{par}
					Dodatkowym elementem może być połączenie kilku metod selekcji celem otrzymania najbardziej optymalnej selekcji dla danego problemu genetycznego. 
					W zasadzie bardziej złożone problemy ewolucyjne wręcz wymagają własnej inwencji do zaprojektowania dobrego systemu.
				\end{par}
		\end{enumerate}
	\end{par}
\end{enumerate}
	Po odpowiednim zaprojektowaniu algorytmu można przystąpić do implementacji, warto jednak pamiętać o przygotowaniu dobrego modułu konfiguracyjnego dla ustawiania poszczególnych elementów, bądź wag w funkcji przystosowania.
	Dużą częścią dobrego systemu genetycznego jest odpowiednia możliwość konfiguracji danych odpowiadających za każdy z kroków.
	Mamy dzięki temu możliwość przetestowania różnych podejść do danego problemu bez uciążliwych zmian w kodzie. 
	Oprócz tego cały proces można zautomatyzować, dzięki czemu możemy w prosty sposób przetestować algorytm dla różnych danych konfiguracyjnych.
\end{par}
