\subsection{Struktura danych chromosomu}
\begin{par}
	Jeśli chodzi o typ struktury danych przyjęto pierwszy schemat (rys.\ref{fig:sterowanie}), wraz z krzyżowaniem statystycznym. 
	O ile ukierunkowanie systemu na gry jednego typu może lepiej się sprawdzić jeśli zależy nam
	jedynie na szybkim i optymalnym wyniku, to generalizacja systemu i elastyczna implementacja całego algorytmu da nam lepsze narzędzie nie tylko dydaktyczne, ale też badawcze.
	Przykładem może być popularna gra Galaxian. Sama definicja klawiszy w Galaxian i Super Mario Brothers jest podobna (klawisze kierunkowe + klawisze specjalne),
	jednak schemat poruszania się jest już inny: O ile gry typu Mario Brothers pasują do powyższych założeń o kierunkowości ruchu, to w grze Galaxian już tak nie jest.
	Korzystniej zatem jest traktowanie projektu jako systemu rozwiązującego gry platformowe czasu rzeczywistego na podstawie przyciśnięć klawiszy w czasie.
\end{par}
\begin{par}
	
	Sam chromosom przechowuje następujące dane:
	\begin{itemize}
		\item Tablicę dotyczącą akcji ruchu przechowującą wartości typu wyliczeniowego: UP,DOWN,LEFT,RIGHT.
		\item Tablicę dotyczącą akcji specjalnych: A,B,C,D
		\item Instancję obiektu ResultData przechowującego dane dotyczące wyniku funkcji przystosowania, oraz wartości ustalane po przetestowaniu danego chromosomu takie jak czas który upłynął, rodzaj wyniku, ilość zebranych punktów.
	\end{itemize}
	Oprócz tego każdy Chromosom uzupełnia interfejs Comparable, dzięki obiekty mogą być porównywane ze sobą. Porównanie składa się jedynie z porównania wyniku wartości funkcji przystosowania. Dzięki temu możemy łatwo posortować całe populacje.
	Obiekt ten przechowuje dane na temat wyniku danego chromosomu i danych pomocniczych, które sa uzupełniane po przetestowaniu chromosomu. Oprócz tego chromosom posiada metody pozwalające na mutację zarówno tablicy ruchu jak i tablicy akcji specjalnych.
\end{par}


