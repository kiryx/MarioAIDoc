\begin{par}
	Tematem pracy jest zaprojektowanie i implementacja systemu podejmującego decyzje w prostej grze platformowej czasu rzeczywistego.
	Celem gry jest przejście dwuwymiarowej mapy. Wynik końcowy może zależeć od wielu parametrów - mapa może zawierać elementy dające punkty, jak i elementy prowadzące do natychmiastowego zakończenia gry (z wynikiem pozytywnym bądź negatywnym).
	Samo określenie jakości przejścia można traktować jako funkcję od czasu przejścia gry, ilości zebranych punktów oraz wyniku końcowego.
	Efektem pracy powinien być system pozwalający na rozwiązanie tego typu problemu bazujący na algorytmie genetycznym.
	\newline
	Ogólne wymogi dotyczące systemu:
	\begin{itemize}
		\item
			System powinien składać się z bazowego silnika gry, wzorowanego na rozwiązaniach w klasycznych grach platformowych.
			System powinien pozwalać zarówno na poruszanie się po mapie przez gracza, jak i przejście w tryb treningu populacji, który na podstawie zadanych parametrów symuluje ruchy gracza. Początkowo są to losowe ruchy, które z czasem są optymalizowane przez algorytm genetyczny.
		\item
			Do wyniku końcowego mogą być brane pod uwagę również inne zdarzenia takie jak ilość zebranych obiektów na planszy, bądź 
			pokonani przeciwnicy.
			Funkcja przystosowania zależeć będzie od rożnych czynników, a ustawienie odpowiednich wag może nakierować algorytm na określoną ścieżkę rozwoju.
		\item
			Dodatkowo, moduł pozwalający na łatwą wizualizację postępów posłuży jako dobra warstwa prezentacyjna postępu algorytmu w czasie.
	\end{itemize}
\end{par}

\begin{par}
	Sam pomysł stworzenia sztucznej inteligencji do gry platformowej w czasie rzeczywistym został już wcześniej wielokrotnie powoływany do życia, m.in. jako projekt MarioAI,
	który w chwili obecnej funkcjonuje jako turniej dla programistów. Uczestnicy mogą implementować własne rozwiazania i porównywać wyniki z innymi uczestnikami.
	Samo zgłoszenie składa się z implementacji własnej klasy odpowiedzialnej za podejmowanie decyzji.
	Strona domowa projektu znajduje się pod adresem www.marioai.org.
\end{par}
