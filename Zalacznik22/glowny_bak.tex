\documentclass[oneside,12pt]{wipb}
\usetikzlibrary{mindmap,trees}%dla diagramu Computer science mindmap


\katedra{Systemów Czasu Rzeczywistego}
\typpracy{ %magisterska
           inżynierska
         }
\temat{Samouczący się automat sterujący postacią gracza w prostej grze zręcznościowej.}
\autor{Krzysztof Nowak}
\promotor{dr inż. Marek Tabędzki }
\indeks{75847}
\studia{stacjonarne
        %niestacjonarne
       }
\rokakademicki{2010/2011}
\profil{%magisterskie jednolite
        %magisterskie uzupełniające
        studia I stopnia
        %studia II stopnia
}
\kierunekstudiow{informatyka
                 %matematyka
                }
\specjalnosc{%Inżynieria Oprogramowania
             %Inżynieria Komputerowa
             %Systemy Oprogramowania
             %Metody infnformatyczne w~banknkowości i~finansach
             %Ochrona systemów informatycznych
            }
\zakres{1. zakres I \newline 2. zakres II \newline 3. zakres III}

\hypersetup{ %wpisy w pdf info
pdfauthor={Krzysztof Nowak},
pdftitle={Samouczący się automat sterujący postacią gracza w prostej grze zręcznościowej.},
pdfsubject={},
pdfkeywords={Praca dyplomowa},
pdfpagemode=UseNone,
linkcolor=black,
citecolor=black
} 
\usepackage[section,subsection,subsubsection]{extraplaceins}
\begin{document}
\maketitle
\tableofcontents
\thispagestyle{empty}
\setcounter{page}{0}
\pagestyle{plain}
\chapter{Wst�p teoretyczny}

\section{Historia sztucznej inteligencji.}
\begin{par}
Na pocz�tku lat 40 matematycy i in�ynierowie z o�rodk�w badawczych zacz�li zastanawia� si� nad mo�liwo�ci� stworzenia sztucznego m�zgu.
Pierwsze formalne centrum badawcze pracuj�ce nad zagadnieniem sztucznej inteligencji zosta�o powo�ane do �ycia w 1956 roku w Dartmouth College, 16 lat po wynalezieniu pierwszego programowalnego komputera.
Pocz�tkowo nazywane przedsi�wzi�ciem stworzenia pierwszego ``Elektronicznego m�zgu'' zosta�o powa�nie potraktowane przez �wczesnych naukowc�w, i tworzy�o dobre perspektywy dla ekonomist�w i bankier�w.
Wielu badaczy zapowiedzia�o stworzenie maszyn dor�wnuj�cych inteligencj� ludziom w niespe�na kilka dekad.
Specjalnie na ten cel rz�d ameryka�ski oraz brytyjski przeznaczy�y bud�et rz�du milion�w dolar�w.
\end{par}
\begin{par}
Pierwsze prace nad sztuczn� inteligencj� skupia�y si� na odwzorowaniu realnej pracy ludzkiego m�zgu - sieci neuron�w.
Naukowcy tacy jak Norbert Wiener, Claude Shannon oraz Alan Turing opracowali pierwsze pomys�y stworzenia elektronicznego m�zgu.
Powsta�o poj�cie sieci neuronowych, mocno p�niej rozwijane m.in. przez Marvina Minskiego w jego pracach przez nast�pne 50 lat.
Pierwsze programy skupiaj�ce si� na sztucznej inteligencji w grach powsta�y na pocz�tku lat 50. Christopher Strachey by� autorem pierwszego programu graj�cego w Warcaby.
Pierwszy program szachowy zosta� napisany przez Dietricha Prinza.
W tym samym okresie Alan Turing opublikowa� pierwsze prace dotycz�ce mo�liwo�ci utworzenia maszyny dysponuj�cej ludzk� inteligencj�.
Zdefiniowa� test pozwalaj�cy to zmierzy�, nazywany potem Testem Turinga.
\end{par}
\begin{par}
W ko�cu po wielu latach sta�o si� oczywiste i� symulacja nawet najprostszych mechanizm�w my�lowych jest niezwykle trudna w realizacji, a nawet najszybsze �wczesne komputery nie by�y w stanie wygra� z cz�owiekiem w partii szach�w. 
Ostatecznie dziedzinie sztucznej inteligencji odebrano nieco wiarygodno�ci, a wcze�niej zapowiadane maszyny przerastaj�ce inteligencj� ludzi, trafi�y z powrotem na p�ki science-fiction. W roku 1973, znaczna cz�� funduszy przeznaczonych na rozw�j sztucznej inteligencji zosta�a wstrzymana przez ameryka�ski i brytyjski rz�d.
Niemniej prace nad sztuczn� inteligencj� trwaj� do dzi�, aczkolwiek s� bardziej uszczeg�owione w naturze problem�w kt�rych dotycz�.
\end{par}


\section{Sztuczna w dzisiejszych zastosowaniach.}
\begin{par}
Opr�cz realizacji zada� w dziedzinie kategoryzacji danych, oraz rozpoznawaniu mowy lub obrazu, spora cz�� bada� skupia si� na realizacji system�w podejmuj�cych decyzje w �ci�le okre�lonym �rodowisku gry.
Pozornie s�u�� one jedynie dostarczaniu rozrywki w szeroko popularnych grach komputerowych, szybko mo�na si� przekona� i� wiele takich projekt�w jest p�niej podstaw� do stworzenia bardziej praktycznych system�w. Zmieniaj�c jedynie definicj� �rodowiska okazuje si� i� mo�na te same strategie zastosowa� np. w grze na gie�dzie.
Pojedynek Garriego Kasparowa z programem szachowym Deep Blue przeszed� ju� na sta�e do historii jako pierwsze starcie cz�owieka z maszyn� w dziedzinie intelektu. 
Innym, do�� nowym przyk�adem mo�e by� klaster komputer�w Watson \cite{watson}, kt�ry przez kilka tygodni konkurowa� z czo��wk� graczy teleturnieju Jeopardy, popularnym w USA od 1964 roku.
Obydwa projekty pokona�y swoich ludzkich przeciwnik�w, podnosz�c tym samym nieco nadszarpni�ty wizerunek sztucznej inteligencji.
Wiele wsp�czesnych zastosowa� sztucznej inteligencji zyska�o du�� popularno�� dzi�ki globalnej sieci internet.
Prym wiedzie tutaj firma Google, pocz�tkowo znana jedynie jako tw�rca najpopularniejszej i najdok�adniejszej wyszukiwarki, rozbudowuje baz� swoich aplikacji o translatory,
aplikacje do nawigacji map, czy program graficzny Picasa potrafi�cy zidentyfikowa� na zdj�ciu ludzk� twarz.
Wi�kszo�� wyszukiwarek internetowych wykorzystuje r�nego rodzaju systemu wspomagaj�ce w wyszukiwaniu informacji na temat konkretnej frazy.
Aktywna lista podpowiedzi wy�wietlaj�ca si� po wpisaniu pocz�tku popularnej frazy sta�a si� wr�cz standardem w projektowaniu wyszukiwarki.
Wyszukiwarka coraz cz�ciej poprawia b��dy ortograficzne lub niepoprawnie przeliterowane wyrazy zwi�zane np. z bliskim s�siedztwem liter na klawiaturze.
Innym ciekawym zastosowaniem jest t�umaczenie tekstu z dowolnego j�zyka na inny.
Nie dzia�a to na zasadzie bazy danych przechowuj�cej odpowiedniki s��w w ka�dym z j�zyk�w, lecz wykorzystuje metod� ``uczenia si�'' ca�ych wyra�e� b�d� zwrot�w kt�re s� r�wnorz�dne w obu j�zykach, 
dzi�ki czemu zyskujemy do�� precyzyjne t�umaczenie nawet je�li chodzi o nazwy w�asne, akronimy, odmian� oraz sk�adni�, np.: Przy t�umaczeniu z angielskiego na polski frazy ``United States of America'' uzyskujemy ``Stany Zjednoczone'', natomiast ``in UK'' t�umaczone na j�zyk polski poprawnie daje zwrot ``w Wielkiej Brytanii''.
W pierwszym przyk�adzie naiwna metoda nie omin�aby s�owa ``America'' oraz prawdopodobnie nie zachowa�aby szyku s��w. Wynikiem mog�oby by� w�wczas ``Zjednoczone Stany Ameryki''.
Innym ciekawym zastosowaniem mo�e by� aplikacja Wolfram Alpha \cite{wolfram} stworzona przez Stephena Wolframa - tw�rc� oprogramowania ``Mathematica''.
Podobnie jak wyszukiwarka Google, opiera ona swoje wyniki na frazach wpisywanych do okna wyszukiwarki przez u�ytkownika, jednak jest ona skupiona g��wnie na przetwarzaniu danych matematycznych.
Potrafi interpretowa� wzory wpisane przez u�ytkownika, wy�wietli� wykres w przestrzeni oraz dostarczy� wielu dodatkowych informacji np. znale�� pierwiastki rzeczywiste.
O ile sztuczna inteligencja w �wietle dzisiejszego dost�pu ogromu danych do przetwarzania stwarza du�y potencja� na tworzenie skomplikowanych system�w przetwarzaj�cych wci�� spotykamy
si� z jej rosn�cym zastosowaniem w grach.
Historia informatyki od jej wczesnych pocz�tk�w zwi�zana by�a z grami komputerowymi. 
W roku 1952 Alexander Shafto Douglas opisa� temat komunikacji cz�owiek-komputer w swojej pracy doktoranckiej, oraz stworzy� program graj�cy w popularne ``K�ko i Krzy�yk''.
Do dzi� uznawany jest on za pierwsz� graficzn� gr� komputerow�.
\end{par}

\section{Podstawy algorytm�w w grach.}
\begin{par}
Jednym z podstawowych przyk�ad�w zastosowania sztucznej inteligencji w grach komputerowych s� gry logiczne. 
Podstawowym algorytmem stosowanym w projektowaniu sztucznej inteligencji jest algorytm minmax.
Najprostszym przyk�adem jest gra ``K�ko i krzy�yk'' gdzie przeszukiwane jest tzw. drzewo gry, kolejno sprawdzaj�ce wszystkie mo�liwe stany gry. 
W dowolnym momencie gry mo�emy przeanalizowa� wszystkie mo�liwe posuni�cia ka�dego z graczy, i d��y� do sytuacji gdzie mamy gwarantowany sukces.
Gra mo�e zako�czy� si� remisem, zwyci�stwem gracza A, b�d� zwyci�stwem gracza B.
Optymalizuj�c decyzj� gracza A, musimy kolejno sprawdza� mo�liwe posuni�cia na planszy i reakcje gracza B. 
Dla ka�dego z mo�liwych ruch�w mo�emy w�wczas przeanalizowa� optymaln� strategi� dla gracza B (poniewa� zak�adamy �e do takiej b�dzie on d��y�) i stara� si� znale�� najlepsz� �cie�k� kt�ra prowadzi do zwyci�stwa (b�d� remisu) gracza A.
Nazwa wzi�a si� w�a�nie od minimalizowania strat, oraz maksymalizowania zysk�w podczas analizy �cie�ek w drzewie.
�atwo zauwa�y� i� algorytm taki ma ogromn� z�o�ono�� dla rozbudowanych drzew, i jedynie dla ma�ych gier takich jak k�ko i krzy�yk daje wynik w realnym czasie.
Bez optymalizacji, oraz brania pod uwag� symetrii planszy, daje to drzewo wywo�a� sk�adaj�ce si� 9! w�z��w (��cznie z li��mi). 
Oczywi�cie algorytm mo�na zoptymalizowa� chocia�by poprzez programowanie dynamiczne, lecz dla gier bardziej z�o�onych nie b�dziemy w stanie przeanalizowa� wszystkich mo�liwych sytuacji w grze w realnym czasie.
Przez lata szachy by�y jedn� z gier niemo�liwych do rozwi�zania za pomoc� powy�szego podej�cia.
Nawet dzi� najlepsze programy szachowe nie graj� idealnie - nie analizuj� wszystkich mo�liwych sytuacji w grze, a jedynie kilkadziesi�t ruch�w w prz�d.
Przy zastosowaniu optymalizacji oraz bazy danych zawieraj�cej wiele strategii szachowych, wsp�czesne programy szachowe wygrywaj� z najlepszymi graczami.
Nieco inna sytuacja jest w grze GO, gdzie plansza rozmiaru 19x19 stanowi du�e wyzwanie nawet dla wsp�czesnych superkomputer�w. 
Do dzi� nie stworzono programu kt�ry wygrywa�by z profesjonalnymi zawodnikami GO.
Jak wida� na Rys. \ref{fig:xo_tree}, pierwsze 2 poziomy drzewa gry K�ko i Krzy�yk nie s� zbyt skomplikowane je�li we�miemy pod uwag� symetri�.
Pocz�tkowo z�o�ono�� drzewa ro�nie wyk�adniczo, jednak mo�liwo�ci na planszy ko�cz� si� zanim zaczyna to by� problemem.
Inaczej wygl�da�by przypadek drzewa gry GO (plansza rozmiaru 13x13), gdzie pierwszy ruch mo�na wykona� na 28 unikalnych sposob�w: Rys. \ref{fig:go_tree} przedstawia przyk�adowe pierwsze 28 w�z��w drzewa gry GO (13x13), bior�c pod uwag� symetri� planszy.
Sprawa komplikuje si� gdy rozpatrujemy wi�ksze plansze.
Dla plansz o wymiarach 19x19 - standardowym rozmiarze obowi�zuj�cym na wszystkich turniejach gry GO - pierwszy ruch mo�na wykona� na 55 sposob�w (analogicznie do Rys.\ref{fig:go_tree}). Plansza GO ma 4 osie symetrii, mo�e si� w�wczas okaza� �e odpowied� przeciwnika b�dzie na tyle ``niesymetryczna'', �e ju� na drugim
poziomie drzewa musimy rozpatrzy� wszystkie mo�liwe pola na planszy, czyli 359 (19x19 - 2).

\begin{figure}[!h]
	\centering
	\includegraphics[width=5in]{obrazki/chess_tree.png}
	\caption{Schemat drzewa gry K�ko i Krzy�yk.}
	\label{fig:xo_tree}
\end{figure}

\begin{figure}[!h]
	\centering
	\includegraphics[width=5in]{obrazki/go_tree.png}
	\caption{Przyk�ad pierwszych mo�liwych unikalnych ruch�w w grze GO.}
	\label{fig:go_tree}
\end{figure}

\end{par}
\begin{par}
Je�li chcemy wprowadzi� podzia� gier przydatny przy projektowaniu systemu pierwszym czynnikiem b�dzie typ rozgrywki ze wzgl�du na czas.
Wi�kszo�� gier mo�na podzieli� w�wczas w nast�puj�cy spos�b:
\begin{itemize}
	\item Gry turowe - Gracze naprzemiennie wykonuj� ruchy, przy czym czas na podj�cie decyzji jest relatywnie du�y - od kilku sekund nawet do 1-2 minut.
Wiele nieskomplikowanych gier turowych zosta�o ju� dawno rozwi�zanych przez algorytmy typu minmax, do tego stopnia, �e systemy graj� w nie ju� niemal bezb��dnie. 
W wielu przypadkach przestrze� rozwi�za� jest jednak wci�� zbyt du�a aby zrealizowa� to algorytmem dok�adnym - przyk�adem mo�e by� wy�ej wspomniana gra GO.
	\item Gry czasu rzeczywistego - Gra toczy si� w dynamicznym �rodowisku gry, cz�sto z wieloma obiektami i graczami na raz. Cz�sto czas na podj�cie optymalnej decyzji przez algorytm jest mocno ograniczony - program musi podejmowa� decyzje nawet do 30 razy w ci�gu sekundy. Opr�cz tego pr�ba dyskretyzacji �rodowiska i znalezienia dok�adnego rozwi�zania z przestrzeni stan�w gry jest w praktyce niewykonalna. Przyk�adem mog� by� tutaj r�nego rodzaju dwuwymiarowe lub tr�jwymiarowe gry akcji, kt�re cz�sto posiadaj� z�o�one �rodowiska gry, wiele dynamicznych obiekt�w oraz graczy uczestnicz�cych w rozgrywce poprzez przez sie� komputerow�. Przy projektowaniu sztucznej inteligencji w takim �rodowisku w tej chwili mo�emy liczy� jedynie na wyniki przybli�one.
\end{itemize}
\end{par}
\begin{par}
Sztuczna inteligencja w grach mo�e dotyczy� r�nych aspekt�w gry. 
W grach logicznych (turowych) g��wnym, i jedynym problemem jest podj�cie najlepszej decyzji dla aktualnego stanu gry daj�cej zwyci�stwo. 
W wi�kszo�ci gier logicznych sztuczna inteligencja ma za zadanie symulacj� godnego przeciwnika dla cz�owieka.
Cz�sto jednak te same algorytmy mog� s�u�y� do cel�w edukacyjnych b�d� do podpowiedzi - ten sam system graj�cy w szachy mo�e gra� przeciwko nam, jak i podpowiada� nam ruchy na podstawie naszej pozycji na planszy.
Podobnie wygl�da sytuacja w grach czasu rzeczywistego, jednak problem znacznie si� komplikuje.
Poniewa� przestrze� rozwi�za� jest bardzo du�a, cz�sto podj�cie decyzji mo�e by� wspomagane przez algorytm przybli�ony, b�d� oparty na algorytmach genetycznych.
Algorytm minmax w wi�kszo�ci przypadk�w zawodzi, b�d� jego czas dzia�ania jest zbyt wolny do zastosowania w dynamicznie dzia�aj�cym �rodowisku.
Korzystaj�c z algorytm�w ``jedynie'' optymalizuj�cych rozgrywk� tracimy mo�liwo�� rozegrania idealnej partii gry, jednak cz�sto wystarcza to dla stworzenia godnego przeciwnika dla ludzkich graczy.
\end{par}

\section{Podstawy algorytm�w genetycznych.}
\begin{par}
Cz�sto wykorzystywanym sposobem rozwi�zania z�o�onego problemu algorytmicznego s� algorytmy genetyczne.
Opieraj� si� one na zasadach ewolucji odkrytych przez Charlesa Darwina, i wzoruj� si� na faktycznych rozwi�zaniach doboru naturalnego wyst�puj�cych w przyrodzie.
Algorytm ewolucyjny opiera si� na wprowadzeniu losowego czynnika do ca�ej procedury, i tym te� r�ni si� od klasycznego algorytmu, i� jest niedeterministyczny. 
Og�lny przebieg algorytmu genetycznego mo�e wygl�da� nast�puj�co:
\begin{enumerate}
\item Wygenerowanie pocz�tkowej populacji osobnik�w (propozycji rozwi�za�) w spos�b losowy.
\item Przeliczenie funkcji przystosowania dla ka�dego z osobnik�w.
\item Uporz�dkowanie populacji malej�co wzgl�dem wyniku funkcji przystosowania.
\item Wybranie populacji rodzicielskiej zgodnie z przyj�t� metod� selekcji.
\item Krzy�owanie osobnik�w z populacji rodzicielskiej i otrzymanie nowej populacji - nowe potomstwo posiada cechy rodzic�w kt�rzy w poprzedniej populacji byli najlepiej przystosowani do rozwi�zania danego problemu.
\item Mutacja cz�ci potomstwa - wprowadzenie czynnika losowego poprzez zmian� niekt�rych fragment�w chromosomu w spos�b losowy.
\item Je�li warunek ko�cowy nie zosta� osi�gni�ty powr�t do kroku 2, w przeciwnym wypadku koniec algorytmu.
\end{enumerate}
Wynikiem takiego algorytmu jest nie jedno rozwi�zanie problemu, a ca�a ich populacja.
W wi�kszo�ci algorytm�w genetycznych mo�na wydzieli� kilka koniecznych do zaprojektowania klas b�d� procedur.
\begin{enumerate}
\item Chromosom oraz Populacja
	\begin{par}
		Pierwszym krokiem jest zdefiniowanie typu danych odpowiednich do przetrzymywania informacji o danym osobniku.
		Odpowiednio zaprojektowany format danych (zwany Chromosomem) pozwoli na �atw� implementacj� pozosta�ych element�w oraz zapewni generowanie optymalnych wynik�w.
		Informacja ta cz�sto jest reprezentowana przez tablic� warto�ci, b�d� list� cech przypisanych do danej klasy. 
		Chromosom odpowiada za informacj� o pojedynczym osobniku, natomiast Populacja traktowana jest jako wszystkie osobniki nale��ce do danego zbioru w danej iteracji algorytmu. 
		O ile w podstawowych algorytmach genetycznych Populacja jest jedynie kontenerem, dobrze jest pami�ta� o ewentualnym rozbudowaniu Populacji do bardziej z�o�onej klasy, dzi�ki czemu b�dziemy mieli mo�liwo�� prostego por�wnywania, b�d� zapami�tywania ca�ych populacji.
	\end{par}
\item Funkcja Przystosowania
	\begin{par}
		Kolejnym istotnym krokiem jest zdefiniowanie funkcji przystosowania.
		W doborze naturalnym wyst�puj�cym w przyrodzie, osobniki danego gatunku ro�liny b�d� zwierz�cia r�ni� si� pod wzgl�dem genetycznym. 
		Mo�na w�wczas wywnioskowa� i� cz�� z nich jest lepiej przystosowana do danego �rodowiska, co z kolei wp�ywa na ich szanse prze�ycia w trudnych sytuacjach, liczno�� potomstwa, d�ugo�� �ycia.
		Poniewa� potomstwo dziedziczy geny po swoich rodzicach, ``zwyci�skie'' cechy w kolejnym pokoleniu s� bardziej powszechne.
		Odpowiednikiem funkcji przystosowania jest w�a�nie wynikowa cech danego osobnika kt�ra okre�la prawdopodobie�stwo przekazania jego gen�w w kolejnym pokoleniu.
		Funkcja przystosowania jest do�� prosta w realizacji, o ile dane dotycz�ce osobnika s� �atwe do zmierzenia -- w�wczas mo�e by� to jedynie kwestia policzenia warto�ci funkcji liniowej z odpowiednimi wagami, gdzie argumentami s� wyniki osobnika podczas symulacji w �rodowisku.
		Mimo to w wi�kszo�ci algorytm�w genetycznych dobranie odpowiednich wag w funkcji przystosowania jest kluczowym czynnikiem nad kt�rym p�niej mo�na d�ugo pracowa� przy optymalizacji algorytmu.
	\end{par}
\item Krzy�owanie
	\begin{par}
		Po ka�dym kroku algorytmu zazwyczaj mo�emy uporz�dkowa� osobniki nale��ce do bie��cej populacji i wylosowa� z niej pewien zbi�r osobnik�w najlepiej przystosowanych (wp�yw na to wynik funkcji przystosowania). 
		W�wczas dokonujemy krzy�owania pomi�dzy nimi, dzi�ki czemu otrzymujemy osobniki nowe, jednak posiadaj�ce pewne cechy swoich ``rodzic�w''.
		Krok ten jest kluczowy je�li chcemy osi�ga� coraz lepsze wyniki w kolejnych populacjach, poniewa� od dobrej metody krzy�owania zale�y czy kolejne populacje b�d� lepiej przystosowane do rozwi�zania problemu.
		Z�e zaprojektowanie krzy�owania jest jednym z cz�stszych powod�w osi�gania przez populacj� z�ych wynik�w, zw�aszcza gdy Chromosom ma z�o�on� struktur�.
		Samo krzy�owanie cz�sto r�wnie� posiada czynnik losowy (w klasycznych przyk�adach dotycz�cych krzy�owania si� dw�ch ci�g�w bitowych, losowany jest punkt ��czenia si� dw�ch ci�g�w).
	\end{par}
\item Mutacja
	\begin{par}
		O ile pocz�tkowa losowo�� algorytmu polegaj�ca na wylosowaniu pierwszej populacji jest szybko zast�powana przez populacj� osi�gaj�c� lepsze wyniki, 
		warto w trakcie ca�ego procesu pr�bowa� modyfikowa� kilka osobnik�w, nawet je�li mog�oby to spowodowa� chwilowe pogorszenie populacji. 
		W innym przypadku zbyt uporz�dkowana procedura selekcji i krzy�owania osobnik�w spowoduje stagnacj� populacji.
		Cz�sto mo�na to zauwa�y� gdy po kilku iteracjach wi�kszo��, b�d� ca�a populacja jest identyczna.
		Najcz�stsz� realizacj� mutacji jest zmiana jakiego� parametru (b�d� grupy parametr�w) danego osobnika na warto�� zupe�nie losow�.
		Poniewa� w du�ej mierze zale�y to od budowy Chromosomu, nie ma uniwersalnej metody na zaimplementowanie mutacji.
		Najcz�ciej mutacja wyst�puje z niskim prawdopodobie�stwem,
		\begin{center}
			$p_m < 0.1$
		\end{center}
		tak aby nie ingerowa� zbyt mocno w algorytm. Ostatecznie nale�y d��y� do pewnej systematycznej optymalizacji, a nie tylko polega� na czynniku losowym.
	\end{par}
\item Metoda Selekcji
	\begin{par}
		Sama metoda wyboru populacji rodzicielskiej r�wnie� ma znaczenie, poniewa� jednak jest ona oparta na warto�ci funkcji przystosowania, to ju� sama metoda wyboru ma mniej krytyczne znaczenie.
		Najbardziej popularne metody selekcji to:
		\begin{enumerate}
			\item Metoda ko�a ruletki.
				\begin{par}
					Sama nazwa bierze si� od popularnej gry w ruletk�, w kt�rej pole powierzchni ka�dego wycinka ko�a jest proporcjonalne do prawdopodobie�stwa wylosowania danej liczby. 
					Oczywi�cie w klasycznej ruletce pola wycink�w ko�a s� r�wne, zatem szansa wylosowania ka�dej liczby jest taka sama.
					W samym algorytmie wirtualne ``wycinki ko�a'' nie musz� oczywi�cie by� r�wne. 
					Osobnik kt�ry osi�ga lepsze wyniki w funkcji przystosowania otrzymuje wi�ksze prawdopodobie�stwo w��czenia do populacji rodzicielskiej ni� osobniki s�absze. 
					Aby to zrealizowa� losowana jest pewna warto�� (najcz�ciej z przedzia�u [0,1], liczb wymiernych) kt�ra potem jednoznacznie okre�la kt�ry osobnik zosta� wylosowany.
					Praktycznie realizowane jest to w nast�puj�cy spos�b:
					\begin{center}
						$p(k)=\frac{f(k)}{\displaystyle\sum\limits_{i=0}^n f(i)}$
					\end{center}
					gdzie $p(k)$ oznacza prawdopodobie�stwo wylosowana k-tego osobnika z populacji, a $f(i)$ warto�� funkcji przystosowania i-tego osobnika. Poniewa� warto�ci s� znormalizowane, suma prawdopodobie�stw wylosowania ka�dego z osobnik�w jest r�wna 1.
					Po uporz�dkowaniu osobnik�w, potrzebna jest jedynie losowa warto�� ktora jednoznacznie okre�li wyb�r osobnika.
					
				\end{par}
			\item Metoda rankingowa.
				\begin{par}
					W tej metodzie sortujemy osobniki malej�co wzgl�dem funkcji przystosowania i wybieramy populacj� rodzic�w jako $m$ pierwszych osobnik�w (zwan� cz�sto elit� populacji). 
					Ma to pewn� wad�, gdy� powoduje po pewnym czasie stagnacj� (brak czynnika losowego). 
					Innym wariantem jest selekcja turniejowa w kt�rej najpierw dzielimy grup� na $G$ podgrup spo�r�d kt�rych wybieramy najlepsze osobniki do populacji rodzicielskiej. 
					Otrzymujemy w ten spos�b G rodzic�w, w�r�d kt�rych niekoniecznie s� najlepsze osobniki globalnie (nawet z bardzo silnej grupy przechodzi tylko jeden osobnik). 
					Daje nam to ju� pewn� losowo�� w wyborze populacji rodzicielskiej.
				\end{par}
			\item Po��czenie kilku metod.
				\begin{par}
					Dodatkowym elementem mo�e by� po��czenie kilku metod selekcji celem otrzymania najbardziej optymalnej selekcji dla danego problemu genetycznego. 
					W zasadzie bardziej z�o�one problemy ewolucyjne wr�cz wymagaj� w�asnej inwencji do zaprojektowania dobrego systemu.
				\end{par}
		\end{enumerate}
	\end{par}
\end{enumerate}
	Du�� cz�ci� dobrego systemu genetycznego jest odpowiednia mo�liwo�� konfiguracji danych odpowiadaj�cych za ka�dy z krok�w.
	Mamy dzi�ki temu mo�liwo�� przetestowania r�nych podej�� do danego problemu bez bezpo�rednich i cz�sto uci��liwych zmian w kodzie programu.
	Opr�cz tego ca�y proces mo�na zautomatyzowa�, dzi�ki czemu mo�emy w prosty spos�b przetestowa� algorytm dla r�nych danych konfiguracyjnych.
\end{par}

\input{analiza2.tex}
\input{interfejs2.tex}
\input{podsumowanie2.tex}
\nocite{*} %wszystkie wpisy w bibliografi
\bibliographystyle{unsrt} %{latex8} posortowane wzgledem wystepowania
\bibliography{bibliografia}%

\addtocontents{toc}{\contentsline {chapter}{Bibliografia}{\thepage}{}}
\listoftables
\addtocontents{toc}{\contentsline {chapter}{Spis tabel}{\thepage}{}}
\listoffigures
\addtocontents{toc}{\contentsline {chapter}{Spis rysunków}{\thepage}{}}
\lstlistoflistings
\addtocontents{toc}{\contentsline {chapter}{Spis listingów}{\thepage}{}}
\listofalgorithms % w zaleznosci od kompilatora i wersji klasy moga wystapic bledy przy kompilacji
\addtocontents{toc}{\contentsline {chapter}{Spis algorytmów}{\thepage}{}}

%\biblioteka{tak} % tak/nie
\end{document}
